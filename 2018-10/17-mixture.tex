\resetproblem \begingroup % \jeolmheader
    \def\jeolmdate{17 октября 2018 г.}% среда
    \def\jeolmauthors{Орлов О.\,П., Тихонов Ю.\,В.}%
\jeolmheader \endgroup

\worksheet{Разнобой}

\begin{problems}

\item 
При любых действительных $a$, $b$, $c$ докажите неравенство
\[
    a^2 b^2 + b^2 c^2 + c^2 a^2 \geq abc(a + b + c)
\, . \]

\item 
На~доске $8 \times 8$ отмечено $16$ клеток, причём в~каждой строке и~в~каждом столбце отмечено ровно по~две клетки.
Докажите, что в~отмеченные клетки можно поставить $8$ чёрных и~$8$ белых ладей так, чтобы в~каждой строке и~в~каждом столбце стояла одна чёрная ладья и~одна белая ладья.

\item 
В~остроугольном треугольнике $ABC$ проведены высоты $AA_1$, $BB_1$, $CC_1$.
Докажите, что точка, симметричная точке $A_1$ относительно стороны~$AB$, лежит на~прямой~$B_1C_1$.

\item 
В~таблицу $101 \times 101$ поставлена $101$ ладья, причём ладьи не~бьют друг друга.
Каждая ладья сделала ход конём.
Могла~ли получиться расстановка ладей, вновь не~бьющих друг друга?

\item 
В~параллелограмме $ABCD$ диагональ $AC$ длиннее диагонали $BD$.
На~диагонали $AC$ отмечена такая точка $M$, что четырёхугольник $MBCD$~--- вписанный.
Докажите, что $BD$ является общей касательной окружностей, описанных вокруг треугольников $ABM$ и~$AMD$.

\item 
Мишень <<бегущий кабан>> находится в~одном из~$100$ окошек, расположенных в~ряд.
Окошки закрыты занавесками так, что для стрелка мишень все время остается невидимой.
Чтобы поразить мишень, достаточно выстрелить в~окошко, в~котором она в~момент выстрела находится.
Если мишень не~была поражена, то~сразу после выстрела она обязательно перемещается в~какое-то соседнее окошко.
Постройте алгоритм, по~которому надо стрелять, чтобы наверняка поразить мишень.

\item 
Докажите, что любое \emph{целое} число можно представить в~виде суммы пяти кубов \emph{целых} чисел.

\itemx{$^+$}
Пусть $f(x) = x^3 - x$, $g(x) = x^3 - 3 x^2 + 1$.
Докажите, что при любых действительных $\alpha$ и $\beta$, сумма которых не равна 0, многочлен $\alpha f(x) + \beta g(x)$ имеет три различных действительных корня.

\end{problems}

