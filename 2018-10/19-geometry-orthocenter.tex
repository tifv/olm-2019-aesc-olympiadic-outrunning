\resetproblem \begingroup % \jeolmheader
    \def\jeolmdate{19 октября 2018 г.}% понедельник
    \def\jeolmauthors{Доледёнок~А.\,В., Орлов~О.\,П.}%
\jeolmheader \endgroup

\worksheet{Ортоцентр}

\noindent
В задачах листика (кроме $6$-й) треугольник $ABC$ --- остроугольный и неравнобедренный; $AA_1$, $BB_1$, $CC_1$ --- его высоты, $H$ --- ортоцентр (точка пересечения высот $AA_1$, $BB_1$, $CC_1$), $O$ --- центр описанной окружности треугольника $ABC$. \\

\begin{problems}

\item
Докажите, что $H$ является точкой пересечения биссектрис треугольника $A_1B_1C_1$.

\item
\subproblem
Докажите, что $\angle{CAO} = \angle{BAH}$. \\
\subproblem
Докажите, что $OA \perp B_1C_1$.

\item 
\subproblem 
Докажите, что точка, симметричная точке $H$ относительно стороны треугольника $ABC$, лежит на описанной окружности этого треугольника. \\ 
\subproblem
Докажите, что точка $X$, симметричная точке $H$ относительно середины стороны $BC$, лежит на описанной окружности треугольника $ABC$, причём отрезок $AX$ является её диаметром.

\item 
Докажите, что расстояние от точки $O$ до стороны $BC$ вдвое меньше длины отрезка $AH$.

\item
Окружность, описанная вокруг треугольника $AB_1C_1$, вторично пересекает описанную окружность треугольника $ABC$ в точке $K$. Докажите, что прямая $KH$ делит сторону $BC$ пополам.

\item
Точки $K$, $L$, $M$, $N$ --- середины сторон $AB$, $BC$, $CD$, $AD$ соответственно вписанного четырёхугольника $ABCD$. Докажите, что ортоцентры треугольников $AKN$, $BKL$, $CML$ и $DMN$ образуют параллелограмм.

\item
Прямые $AO$ и $BC$ пересекаются в точке $Q$, а отрезки $AH$ и $B_1C_1$~--- в точке $P$. Точка $M$~--- середина $BC$. Докажите, что $HM \parallel PQ$.

\end{problems}


