\resetproblem \begingroup % \jeolmheader
    \def\jeolmdate{7 ноября 2018 г.}% понедельник
    \def\jeolmauthors{Орлов~О.\,П., Тихонов~Ю.\,В.}%
\jeolmheader \endgroup

\worksheet{Графы. Связность}

\claim{Определение} \textit{Граф} задан, если задано множество его \textit{вершин} и для любой пары различных вершин известно, связаны они \textit{ребром} или нет. \textit{Степенью} вершины называется количество выходящих из неё рёбер. Вершина называется \textit{чётной}, если её степень чётна и \textit{нечётной} --- если её степень нечётна. Степень вершины $v$ обозначается $\deg{v}$.

\claim{Утверждение} Количество рёбер в графе есть полусумма степеней всех его вершин.

\claim{Определение} \textit{Путём} в графе называется последовательность рёбер, каждое следующее из которых начинается в конце предыдущего. \textit{Циклом} в графе называется замкнутый путь, то есть такой, начало и конец которого совпадают. 

\claim{Определение} Граф называется \textit{связным}, если любые две его вершины могут быть соединены путём. Любой граф является объединением некоторых связных <<кусков>>, каждый такой <<кусок>> называется \textit{компонентой связности}. 

\begin{problems}

\item В стране Семерка $15$ городов, каждый из которых соединен дорогами не менее, чем с $7$ другими. Докажите, что из любого города можно добраться до любого другого (возможно, проезжая через другие города).

\item В Тридевятом царстве лишь один вид транспорта -- ковёр-самолёт. Из столицы выходит $21$ ковролиния, из города Дальний - одна, а из всех остальных городов -- по $20$. Докажите, что из столицы можно долететь в Дальний (возможно, с пересадками).

\item Между некоторыми из $2n$ городов установлено воздушное сообщение, причём каждый город связан (беспосадочными рейсами) не менее чем с $n$ другими. Докажите, что если отменить любые $n-1$ рейсов, то всё равно из любого города можно добраться в любой другой на самолётах (с пересадками).

\item Столица страны соединена авиалиниями со $100$ городами, а каждый город, кроме столицы, соединён авиалиниями ровно с $10$ городами (если $A$ соединён с $B$, то $B$ соединён с $A$). Известно, что из любого города можно попасть в любой другой (может быть, с пересадками). Докажите, что можно закрыть половину авиалиний, идущих из столицы так, что возможность попасть из любого города в любой другой сохранится.

\item Гидры состоят из голов и шей (любая шея соединяет ровно две головы). Одним ударом меча можно снести все шеи, выходящие из какой-то головы $A$ гидры. Но при этом из головы $A$ мгновенно вырастает по одной шее во все головы, с которыми $A$ не была соединена. Геракл побеждает гидру, если ему удастся разрубить её на
две несвязанные шеями части. Найдите наименьшее $N$, при котором Геракл сможет победить любую  \textit{стошеюю} гидру, нанеся не более, чем $N$ ударов.

\item В стране $1001$ город, любые два города соединены дорогой с односторонним движением (граф в котором на каждом ребре задано направление называется \textit{ориентированным}). Из каждого города выходит ровно $500$ дорог, в каждый город входит ровно $500$ дорог. От страны отделилась независимая республика, в которую вошли $668$ городов. Докажите, что из любого города этой республики можно доехать до любого другого её города, не выезжая за пределы республики.

\end{problems}


