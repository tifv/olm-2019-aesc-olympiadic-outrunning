\resetproblem \begingroup % \jeolmheader
    \def\jeolmdate{16 ноября 2018 г.}% среда
    \def\jeolmauthors{Доледенок А.\,В., Орлов О.\,П.}%
\jeolmheader \endgroup

\worksheet{Комбинаторная геометрия. Принцип крайнего}

\begin{problems}


\item Из точки $O$ выходит несколько лучей так, что угол между любыми двумя меньше $120^{\circ}$. Докажите, что среди этих лучей найдутся два луча, в угле между которыми находятся все остальные.
% Два луча с наибольшим углом между ними являются искомыми. Другой луч не может лежать в соседнем угле по построению, и не может лежать в вертикальном угле, иначе сумма углов между тремя рассмотренными лучами меньше 360.

\item Несколько прямых общего положения делят плоскость на части (никакие две прямые не параллельны, никакие три не пересекаются в одной точке). Докажите, что к каждой прямой примыкает хотя бы один треугольник.
% Возьмем ближайшую к рассматриваемой прямой точку пересечения других прямых $l_1$ и $l_2$. Эти три прямые образуют треугольник, примыкающий к рассматриваемой прямой. Никакая другая прямая не пересекает этот треугольник, так как тогда она бы пересекала хотя бы одну сторону треугольника, лежащей на $l_1$ или $l_2$, что противоречило бы выбору ближайшей точки пересечения прямых к рассматриваемой.

\item Несколько прямых общего положения делят плоскость на части. Докажите, что хотя бы одна часть является углом.
%б) Докажите, что хотя бы три части являются углами.
% Принцип крайнего по координате.
% В пункте а) достаточно взять самую левую точку пересечения прямых в системе координат, в которой оси не параллельны ни одной из данных прямых.
% Пункт б). Рассмотрим в такой системе координат самую левую, самую правую, самую верхнюю и самую нижнюю точки. Если из рассмотренных хотя бы три различных, то для каждой мы найдем свой угол и мы победили. Пусть две, от каждой из них выходит по углу. Соединим эти точки прямой $l$ и рассмотрим самую дальную от этой прямой точку пересечения данных прямых, если их несколько то возьмём из них крайнюю на прямой, их содержащей. Она и будет точкой, из которой исходит третий искомый угол. 

\item Дан выпуклый многоугольник площади $1$. \\
а) Докажите, что существует прямоугольник площади $2$, в котором данный многоугольник полностью содержится; \\
б) Докажите, что существует прямоугольник площади $1/8$, который целиком содержится в данном многоугольнике.
% Рассмотрим две вершины многоугольника $A$ и $B$ на наибольшем расстоянии. Тогда весь многоугольник лежит в полосе, в которой $AB$ --- высота. Пусть полоса находится вертикально. Возьмем самую высокую вершину многоугольника $C$ (если их несколько --- то любую из них) и самую низкую --- $D$ ($C$ или $D$ может совпадать с $A$ или $B$). Прямые, параллельные $AB$, проходящие через $C$ и $D$ ограничивают полосу до прямоугольника $P$.
% Пункт а). Прямоугольник $P$ искомый, так как $S_P = 2S_{ABC} + 2S_{ABD} \leqslant 2S_{многоуг} = 2.
% Пункт б). Заметим, что $S_P \geqslant S_{многоуг} = 1$. Прямая $AB$ разрезает $P$ на два более маленьких прямоугольника. Без ограничения общности пусть площадь верхнего, содержащего точку $C$, не меньше половины $P$, а значит не меньше $1/2$. Тогда $S_{ABC} \geqslant 1/4$. А в треугольник $ABC$ можно вписать прямоугольник $p$, площадь которого не меньше половины площади $ABC$ ($p$ содержит среднюю линюю треугольника).

\item Дан треугольник со сторонами, не превосходящими $1$. Докажите, что круги радиуса $1/\sqrt{3}$ с центрами в вершинах треугольника полностью его покрывают.
% ABC --- треугольник. Пусть точка $X$ не покрыта. Тогда $XA, XB, XC \geqslant 1 / \sqrt{3}$. Хотя бы один из углов $AXB, BXC, CXA$ не меньше 120, пусть $AXB$. Но тогда по теореме косинусов $AB > 1$, противоречие.

\item На плоскости отмечено $N \geqslant 3$ различных точек. Известно, что среди попарных расстояний между отмеченными точками встречаются не более $n$ различных расстояний. Докажите, что $N \leqslant (n + 1)^2$.
% Найдём такие точки $AB$, что все остальные лежат по одну сторону от $AB$ (возможно на $AB$ какие-то). Для этого рассмотрим самую левую точку $A$ и повращаем вертикальную прямую вокруг $A$, пока она не наткнётся на искомую точку $B$. Отмеченные точки лежат на $n$ концентрических окружностях с центрами в $A$ и на $n$ концентрических окружностях с центрами в $B$. Точек пересечения всех этих окружностей не больше $2n^2$ и не более $n^2$ в одной полуплоскости относительно $AB$. Поэтому всего отмеченных точек не больше $n^2 + 2 \leqslant (n + 1)^2$.

\item На плоскости отмечено несколько точек, причём на каждой прямой, проходящей через любые две отмеченные точки, есть ещё хотя бы одна отмеченная точка. Докажите, что все отмеченные точки лежат на одной прямой.
% Рассмотрим все пары $(P, l)$, где $P$ --- отмеченная точка, которая не лежит на прямой $l$, соединяющей две какие-то другие отмеченные точки. Пусть это множество не пусто, возьмем тогда пару с наименьшим расстоянием между элементами пары. По условию на $l$ лежит хотя бы три отмеченных точки $A, B, C$. Опустим высоту $PH$ на $l$. Тогда хотя бы две точки из $A, B, C$, по одну сторону относительно $H$, пусть это точки $A$ и $B$, причём точка $A$ лежит дальше от $H$. Тогда, очевидно, расстояние от точки $B$ до прямой $AP$ меньше, чем $PH$, противоречие.


\end{problems}

