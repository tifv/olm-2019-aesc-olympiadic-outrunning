\resetproblem \begingroup % \jeolmheader
    \def\jeolmdate{9 ноября 2018 г.}% понедельник
    \def\jeolmauthors{Орлов~О.\,П.}%
\jeolmheader \endgroup

\worksheet{Графы. Деревья}

\claim{Определение} \textit{Деревом} называется связный граф без циклов. \textit{Висячей} вершиной в графе называется вершина степени $1$.

\claim{Утверждение} В любом дереве с $n \geqslant 2$ вершинами есть как минимум две висячие вершины.

\claim{Утверждение} В любом дереве с $n$ вершинами ровно $n-1$ ребро.

\claim{Определение} \textit{Остовным деревом} графа называется \textit{подграф}, содержащий все его вершины и являющийся деревом.

\claim{Утверждение} В любом связном графе можно выделить остовное дерево.

\claim{Утверждение} Если в связном графе с $n$ вершинами $n-1$ ребро, то это дерево.

\claim{Утверждение} В любом связном графе с $n$ вершинами не меньше $n-1$ ребер.

\begin{problems}

\item Докажите, что из любого связного графа можно убрать одну вершину вместе со всеми выходящими из неё рёбрами так, чтобы граф остался связным.

\item Куб $n \times n \times n$ разбит на кубики $1 \times 1 \times 1$. Какое минимальное количество граней $1 \times 1$ необходимо в нём убрать, чтобы из любой его части можно было пробраться наружу?

\item В стране $100$ городов, некоторые из которых соединены авиалиниями. Известно, что от любого города можно долететь до любого другого (возможно, с пересадками). Докажите, что можно пролететь по всем городам, сделав при этом не более $196$ перелётов.

\item Даны натуральные взаимно простые числа $p$ и $q$. Вася не знает сколько человек придёт к нему на день рождения, либо $p$, либо $q$. На какое минимальное количество кусков (не обязательно равных) ему нужно заранее разрезать торт, чтобы он мог раздать торт поровну как на $p$ человек, так и на $q$?

\item Петя поставил на доску $50\times 50$ несколько фишек, в каждую клетку -- не больше одной. Докажите, что Вася может поставить на свободные поля этой же доски не более $99$ новых фишек (возможно, ни одной) так, чтобы по-прежнему в каждой клетке стояло не больше одной фишки, и в каждой строке и каждом столбце этой доски оказалось чётное количество фишек.

\item Все $n$ вершин графа $G$ занумерованы. Известно, что множество вершин графа можно разбить на пять (потенциально пустых) долей с условием, чтобы внутри этих долей не было рёбер, причём это можно сделать \textit{единственным} с точностью до перестановки долей образом. Докажите, что в графе хотя бы $4n - 10$ рёбер.

\item Дано дерево с $n$ вершинами. В его вершинах расставлены числа $x_1, x_2, \ldots, x_n$, а на каждом ребре записано произведение чисел, стоящих в концах этого ребра. Обозначим через $S$ сумму чисел на всех рёбрах. Докажите, что $\sqrt{n-1}(x_1^2 + x_2^2 + \ldots + x_n^2) \geqslant 2S$.

\end{problems}


