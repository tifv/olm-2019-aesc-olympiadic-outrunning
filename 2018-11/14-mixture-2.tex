\resetproblem \begingroup % \jeolmheader
    \def\jeolmdate{14 ноября 2018 г.}% среда
    \def\jeolmauthors{Орлов О.\,П., Тихонов Ю.\,В.}%
\jeolmheader \endgroup

\worksheet{Разнобой-2}

\begin{problems}

\item Уравнение $a x^5 + b x + c = 0$ имеет ровно три различных корня ($c \neq 0$). Докажите, что уравнение $c x^5 + b x^4 + a = 0$ тоже имеет ровно три различных корня.

\item Положительные числа $a, b, c$ таковы, что $ab + bc + ac \geqslant a + b + c$. Докажите, что $a + b + c \geqslant 3$.

\item Для вписанного четырехугольника выполнено равенство $CD = AD + BC$. Докажите, что точка пересечения биссектрис углов $A$ и $B$ лежит на прямой $CD$.

\item Петя и Вася играют в очень странные шашки на полоске $1 \times n$. Изначально в последних (самых правых) трёх клетках полоски стоит три фишки (по одной в клетке). За ход игрок берёт любую фишку и передвигает её налево в любую свободную клетку (ничего не мешает фишкам перепрыгивать друг друга). Проигрывает тот, кто не может сделать ход. Кто выигрывает при правильной игре, если Петя начинает первым?

\item В графе у любых двух смежных вершин ровно $5$ общих смежных вершин. Докажите, что количество рёбер в этом графе делится на $3$.

\item Дано $10$ натуральных чисел. Докажите, что можно выбрать несколько из них так, чтобы сумма выбранных делилась на $10$.

\item Дано натуральное число $n > 2$. Рассмотрим все покраски клеток доски $n \times n$ в $k$ цветов такие, что каждая клетка покрашена ровно в один цвет, и все $k$ цветов встречаются. При каком наименьшем $k$ в любой такой покраске найдутся четыре окрашенных в четыре разных цвета клетки, расположенные в пересечении двух строк и двух столбцов?

\item
Найдите углы остроугольного треугольника $ABC$, если известно, что его
биссектриса~$AD$ равна стороне~$AC$ и~перпендикулярна отрезку~$OH$, где $O$~---
центр описанной окружности, $H$~--- точка пересечения высот треугольника $ABC$.







\end{problems}

