\resetproblem \begingroup % \jeolmheader
    \def\jeolmdate{16 ноября 2018 г.}% среда
    \def\jeolmauthors{Доледенок А.\,В., Орлов О.\,П.}%
\jeolmheader \endgroup

\worksheet{Степень точки. Радикальные оси}

\claim{Определение} \textit{Степенью точки} $P$ относительно окружности $\omega$ с радиусом $r$ называется величина $d^2 - r^2$, где $d$ --- расстояние от точки $P$ до центра окружности.

\claim{Утверждение 1} Пусть прямая, проходящая через точку $P$, пересекает окружность $\omega$ в точках $A$ и $B$. Тогда степень точки $P$ относительно окружности $\omega$ равна числу $PA \cdot PB$, если точка $P$ лежит вне окружности $\omega$, и числу $-PA \cdot PB$ --- если внутри.

\claim{Утверждение 2} Пусть точка $P$ лежит вне окружности $\omega$ и прямая $PX$ касается окружности $\omega$ в точке $X$. Тогда степень точки $P$ относительно окружности $\omega$ равна $PX^2$.

\claim{Утверждение 3} Пусть даны две неконцентрические окружности. Геометрическим местом точек, степени которых относительно этих окружностей равны, является прямая, перпендикулярная линии центров окружностей. Эта прямая называется \textit{радикальной осью} этих двух окружностей. 



\begin{problems}

\item К двум непересекающимся окружностям проведены все четыре общих касательных. Докажите, что середины этих касательных лежат на одной прямой.

\item В четырехугольнике $ABCD$ углы $A$  и $C$ --- прямые. На сторонах $AB$ и $CD$ как на диаметрах построены  окружности,  пересекающиеся  в  точках $X$ и $Y$.  Докажите,  что  прямая $XY$ проходит через середину $K$ диагонали $AC$.

\item На окружности $\omega_1$ с диаметром $AB$ взята точка $C$. Из точки $C$ опущен перпендикуляр $CH$ на диаметр $AB$. Проведена окружность $\omega_2$ с центром в точке $C$ и радиусом $CH$. Докажите, что общая хорда окружностей $\omega_1$ и $\omega_2$ делит отрезок $CH$ пополам.

\end{problems}

