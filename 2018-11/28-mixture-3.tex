\resetproblem \begingroup % \jeolmheader
    \def\jeolmdate{28 ноября 2018 г.}% среда
    \def\jeolmauthors{Орлов О.\,П., Тихонов Ю.\,В.}%
\jeolmheader \endgroup

\worksheet{Разнобой-3}

\begin{problems}

\item В ромбе $ABCD$ на стороне $BC$ отмечена точка $P$, а на отрезке $BD$ точки $S$ и $Q$. Оказалось, что точки $P, C, S, Q$ лежат на одной окружности и точки $A, B, P, S$ тоже лежат на одной окружности. Докажите, что точки $A, Q, P$ лежат на одной прямой.

\item Шахматная доска раскрашена в $10$ цветов правильным образом (соседние по стороне клетки
имеют разные цвета), причём все цвета присутствуют. Два цвета называются соседними, если
существуют две соседние клетки этих цветов. Каково наименьшее число пар соседних цветов?

\item
Найдите углы остроугольного треугольника $ABC$, если известно, что его
биссектриса~$AD$ равна стороне~$AC$ и~перпендикулярна отрезку~$OH$, где $O$~---
центр описанной окружности, $H$~--- точка пересечения высот треугольника $ABC$.

\item В графе у любых двух смежных вершин ровно $5$ общих смежных вершин. Докажите, что количество рёбер в этом графе делится на $3$.

\item Числа $a, b, c$ таковы, что уравнение $x^3 + ax^2 + bx + c = 0$ имеет три различных действительных корня. Докажите, что если $-2 \leqslant a + b + c \leqslant 0$, то хотя бы один из этих корней принадлежит отрезку $[0, 2]$.

\item Из Ленинграда в Москву поочерёдно выходят путники, на каждом из которых сидит своя муха.
Все путники идут с постоянными различными скоростями; для любой пары путников известно,
что более медленный из них вышел из Ленинграда раньше, но пришёл в Москву позже. Когда
два путника встречаются, мухи на них меняются местами (никакие три путника не встречаются
вместе одновременно). Докажите, что какая-то муха побывает на всех путниках.



\end{problems}

