\resetproblem \begingroup % \jeolmheader
    \def\jeolmdate{30 ноября 2018 г.}% среда
    \def\jeolmauthors{Доледенок А.\,В., Орлов О.\,П.}%
\jeolmheader \endgroup

\worksheet{Степень точки. Радикальные оси}

\claim{Определение} \textit{Степенью точки} $P$ относительно окружности $\omega$ с радиусом $r$ называется величина $d^2 - r^2$, где $d$ --- расстояние от точки $P$ до центра окружности.

\claim{Утверждение 1} Пусть прямая, проходящая через точку $P$, пересекает окружность $\omega$ в точках $A$ и $B$. Тогда степень точки $P$ относительно окружности $\omega$ равна числу $PA \cdot PB$, если точка $P$ лежит вне окружности $\omega$, и числу $-PA \cdot PB$ --- если внутри.

\claim{Утверждение 2} Пусть точка $P$ лежит вне окружности $\omega$ и прямая $PX$ касается окружности $\omega$ в точке $X$. Тогда степень точки $P$ относительно окружности $\omega$ равна $PX^2$.

\claim{Утверждение 3} Пусть даны две неконцентрические окружности. Геометрическим местом точек, степени которых относительно этих окружностей равны, является прямая, перпендикулярная линии центров окружностей. Эта прямая называется \textit{радикальной осью} этих двух окружностей. 

\claim{Утверждение 4} Даны три окружности, центры которых не лежат на одной прямой. Для каждой пары из этих окружностей проведена радикальная ось. Эти радикальные оси пересекаются в одной точке. Эта точка называется \textit{радикальным центром} данных окружностей.

\begin{problems}

\item К двум непересекающимся окружностям проведены все четыре общих касательных. Докажите, что середины этих касательных лежат на одной прямой.

%\item В четырехугольнике $ABCD$ углы $A$  и $C$ --- прямые. На сторонах $AB$ и $CD$ как на диаметрах построены  окружности,  пересекающиеся  в  точках $X$ и $Y$.  Докажите,  что  прямая $XY$ проходит через середину $K$ диагонали $AC$.

\item На окружности $\omega_1$ с диаметром $AB$ взята точка $C$. Из точки $C$ опущен перпендикуляр $CH$ на диаметр $AB$. Проведена окружность $\omega_2$ с центром в точке $C$ и радиусом $CH$. Докажите, что общая хорда окружностей $\omega_1$ и $\omega_2$ делит отрезок $CH$ пополам.

\item Пусть вписанная окружность треугольника $ABC$ касается сторон $AB$, $AC$, $BC$ в точках $C_1$, $B_1$ , $A_1$. Докажите, что средние линии треугольников $A_1 CB_1$ и $A_1BC_1$ соответственно параллельные сторонам $A_1 B_1$ и $A_1C_1$, а также серединный перпендикуляр к $BC$ пересекаются в одной точке.

\textit{Подсказка.} Точка --- это маленькая окружность с радиусом 0.

\item Дана равнобокая трапеция $ABCD$ с основаниями $AD$ и $BC$. Окружность $\omega$ проходит через точки $A$ и $D$ и пересекает отрезки $AB$ и $AC$ в точках $P$ и $Q$ соответственно. Обозначим через $X$ и $Y$ отражения точек $P$ и $Q$ относительно середин отрезков $AB$ и $AC$ соответственно. Докажите, что точки $B, C, X, Y$ лежат на одной окружности.

\item Дана неравнобокая трапеция $ABCD$ ($AB \parallel CD$). Произвольная окружность, проходящая
через точки $C$ и $D$, пересекает боковые стороны трапеции в точках $P$ и $Q$, а диагонали --- в
точках $M$ и $N$ . Докажите, что прямые $PQ$, $MN$ и $AB$ пересекаются в одной точке.

\item В остроугольном треугольнике $ABC$ проведена высота $CC_1$, продолжение медианы $AM$ пересекает описанную окружность треугольника $ABC$ в точке $N$. Точка $D$ плоскости такова, что $ABCD$ параллелограмм. Докажите, что $A, C_1, N, D$ лежат на одной окружности.

\end{problems}

