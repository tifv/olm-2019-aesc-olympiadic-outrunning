\resetproblem \begingroup % \jeolmheader
    \def\jeolmdate{5 декабря 2018 г.}% среда
    \def\jeolmauthors{Орлов О.\,П., Тихонов Ю.\,В.}%
\jeolmheader \endgroup

\worksheet{Теория чисел}

\begin{problems}

\item Пусть $S(n)$ --- сумма цифр числа $n$. Зададим последовательность $a_{n + 1} = S(a_n)$, $a_0 = 2^{1000000}$. Найдите $a_6$.

\item Сколько простых чисел содержится в последовательности $101, 10101, 1010101, \ldots$?

\item Докажите, что для любого многочлена $P$ с целыми коэффициентами и для любого натурального $k$ существует такое натуральное число $n$, что $P(1) + P(2) + \ldots + P(n)$ делится на $k$.

\item Существует ли такое натуральное число, что, написав его рядом дважды, получится квадрат натурального числа?

\item Через $S(n)$ обозначим сумму цифр числа $n$. Докажите, что не существует натурально-
го $N$ такого, что для любого $n > N$ выполнялось бы неравенство $S(2^n ) \le S(2^{n+1 })$.

\item Докажите, что для любого натурального числа $n$ существует число, делящееся на $n$, состоящее только из нулей и единиц.

\item Пусть $a, b, c, m$ --- такие натуральные числа, что
\[
    a^2 + b^2 + c^2 + 1 = m a b c
\, . \]
Докажите, что $m = 4$.

\end{problems}

