\resetproblem \begingroup % \jeolmheader
    \def\jeolmdate{D MMбря YYYY г.}% понедельник
    \def\jeolmauthors{Пономарев~А.\,А.}%
\jeolmheader \endgroup

\worksheet{Название листика}


\subsection*{Упражнения или разбираемые задачи}

\begin{exercises}

\item
Имеют
\\
\subproblem отдельную нумерацию
\quad
и
\quad
\subproblem отличаются по форматированию.

\end{exercises}


\subsection*{Базовое использование}

(\verb+\subsection+, конечно, вещь совершенно необязательная.)

\begin{problems}

\itemy{0}
Нумерацию отдельных задач можно задавать вручную.

\item
Задача.

\item
\subproblem Пункт;
\\
\subproblemx{*} сложный пункт;
\\
\subproblemy{ы} кривой пункт.

\itemx{*}
Сложная задача.

\end{problems}


\subsection*{Второй раздел}

\claim{Теорема}
Важная, наверное.

\begin{problems}

\item
Нумерация задач продолжается непрерывно.
Команда \verb+\resetproblem+ сбрасывает нумерацию.
Аналогично работает команда \verb+\resetsubproblem+.

\item\claim{Другая теорема}
Но при этом задача.

\end{problems}

Какой-то текст.

\begin{problems}

\item\emph{Ещё теорема.}
Но её название никому всё равно не нужно, так что жирным выделять не стоит.

\item
Текстик\\
\subproblem вариант задачи
\qquad
\subproblem другой вариант
\\
опять текстик.

\end{problems}


\subsection*{Всякие мелочи}

\begin{problems}

\item
В соответствии с Unicode, поменяты местами $\phi$ и $\varphi$, а также
$\epsilon$ и $\varepsilon$.

\item
Определен сравнительно адекватный оператор $a \kratno b$.

\end{problems}

