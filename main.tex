\documentclass[a4paper,10pt]{article}
\usepackage[utf8]{inputenc}

% Конфигурация — общее {{{1

\usepackage{multicol}
\usepackage[svgnames]{xcolor}
\usepackage{graphicx}

\usepackage[lazy-figures]{jeolm}
\usepackage{jeolm-groups}

\usepackage[T2A]{fontenc}
\usepackage{anyfontsize}
\usepackage{amsmath}
\usepackage{amssymb}
\usepackage{upgreek}
\AtBeginDocument{\swapvar{phi}\swapvar{epsilon}}
\AtBeginDocument{\swapvar[up]{phi}\swapvar[up]{epsilon}}
\AtBeginDocument{\let\geq\geqslant\let\leq\leqslant}
\usepackage[russian]{babel}
\usepackage{parskip}
\pagestyle{empty}

% Конфигурация — страницы {{{1

\usepackage{geometry}
% Размер «логической» страницы
\geometry{a5paper,portrait,vmargin={2em,2em},hmargin={2em,2em}}
%\geometry{a6paper,landscape,vmargin={1.5em,1.5em},hmargin={1.5em,1.5em}}

%\usepackage{pgfpages}
% Размещение «логических» страниц на «физической»
%\pgfpagesuselayout{resize to}[a4paper]
%\pgfpagesuselayout{2 on 1}[a4paper,landscape]
%\pgfpagesuselayout{4 on 1}[a4paper,landscape]

\usepackage{hyperref}
\hypersetup{colorlinks,urlcolor=blue}
\def\maybephantomsection{\ifdefined\phantomsection\phantomsection\fi}

% Конфигурация — локальное {{{1

\renewcommand\jeolminstitution
  {СУНЦ МГУ, Олимпиадная математика}%
\renewcommand\jeolmdaterange
  {2018--2019}%
\def\jeolmgroupname{Убегающие}%

% Конфигурация — размер шрифта {{{1

% Уменьшить основной размер шрифта
%\AtBeginDocument{\fontsize{9.00}{10.80}\selectfont}
% (по умолчанию 10.00 и 12.00 соответственно)

% }}}1


\begin{document}

\clearpage\resetproblem \begingroup % \jeolmheader
    \def\jeolmdate{28 ноября 2018 г.}% среда
    \def\jeolmauthors{Орлов О.\,П., Тихонов Ю.\,В.}%
\jeolmheader \endgroup

\worksheet{Разнобой-3}

\begin{problems}

\item В ромбе $ABCD$ на стороне $BC$ отмечена точка $P$, а на отрезке $BD$ точки $S$ и $Q$. Оказалось, что точки $P, C, S, Q$ лежат на одной окружности и точки $A, B, P, S$ тоже лежат на одной окружности. Докажите, что точки $A, Q, P$ лежат на одной прямой.

\item Шахматная доска раскрашена в $10$ цветов правильным образом (соседние по стороне клетки
имеют разные цвета), причём все цвета присутствуют. Два цвета называются соседними, если
существуют две соседние клетки этих цветов. Каково наименьшее число пар соседних цветов?

\item
Найдите углы остроугольного треугольника $ABC$, если известно, что его
биссектриса~$AD$ равна стороне~$AC$ и~перпендикулярна отрезку~$OH$, где $O$~---
центр описанной окружности, $H$~--- точка пересечения высот треугольника $ABC$.

\item В графе у любых двух смежных вершин ровно $5$ общих смежных вершин. Докажите, что количество рёбер в этом графе делится на $3$.

\item Числа $a, b, c$ таковы, что уравнение $x^3 + ax^2 + bx + c = 0$ имеет три различных действительных корня. Докажите, что если $-2 \leqslant a + b + c \leqslant 0$, то хотя бы один из этих корней принадлежит отрезку $[0, 2]$.

\item Из Ленинграда в Москву поочерёдно выходят путники, на каждом из которых сидит своя муха.
Все путники идут с постоянными различными скоростями; для любой пары путников известно,
что более медленный из них вышел из Ленинграда раньше, но пришёл в Москву позже. Когда
два путника встречаются, мухи на них меняются местами (никакие три путника не встречаются
вместе одновременно). Докажите, что какая-то муха побывает на всех путниках.



\end{problems}


\clearpage\resetproblem \begingroup % \jeolmheader
    \def\jeolmdate{28 ноября 2018 г.}% среда
    \def\jeolmauthors{Орлов О.\,П., Тихонов Ю.\,В.}%
\jeolmheader \endgroup

\worksheet{Разнобой-3}

\begin{problems}

\item В ромбе $ABCD$ на стороне $BC$ отмечена точка $P$, а на отрезке $BD$ точки $S$ и $Q$. Оказалось, что точки $P, C, S, Q$ лежат на одной окружности и точки $A, B, P, S$ тоже лежат на одной окружности. Докажите, что точки $A, Q, P$ лежат на одной прямой.

\item Шахматная доска раскрашена в $10$ цветов правильным образом (соседние по стороне клетки
имеют разные цвета), причём все цвета присутствуют. Два цвета называются соседними, если
существуют две соседние клетки этих цветов. Каково наименьшее число пар соседних цветов?

\item
Найдите углы остроугольного треугольника $ABC$, если известно, что его
биссектриса~$AD$ равна стороне~$AC$ и~перпендикулярна отрезку~$OH$, где $O$~---
центр описанной окружности, $H$~--- точка пересечения высот треугольника $ABC$.

\item В графе у любых двух смежных вершин ровно $5$ общих смежных вершин. Докажите, что количество рёбер в этом графе делится на $3$.

\item Числа $a, b, c$ таковы, что уравнение $x^3 + ax^2 + bx + c = 0$ имеет три различных действительных корня. Докажите, что если $-2 \leqslant a + b + c \leqslant 0$, то хотя бы один из этих корней принадлежит отрезку $[0, 2]$.

\item Из Ленинграда в Москву поочерёдно выходят путники, на каждом из которых сидит своя муха.
Все путники идут с постоянными различными скоростями; для любой пары путников известно,
что более медленный из них вышел из Ленинграда раньше, но пришёл в Москву позже. Когда
два путника встречаются, мухи на них меняются местами (никакие три путника не встречаются
вместе одновременно). Докажите, что какая-то муха побывает на всех путниках.



\end{problems}



\end{document}

% Архив {{{1

\clearpage\resetproblem \begingroup % \jeolmheader
    \def\jeolmdate{17 октября 2018 г.}% среда
    \def\jeolmauthors{Орлов О.\,П., Тихонов Ю.\,В.}%
\jeolmheader \endgroup

\worksheet{Разнобой}

\begin{problems}

\item 
При любых действительных $a$, $b$, $c$ докажите неравенство
\[
    a^2 b^2 + b^2 c^2 + c^2 a^2 \geq abc(a + b + c)
\, . \]

\item 
На~доске $8 \times 8$ отмечено $16$ клеток, причём в~каждой строке и~в~каждом столбце отмечено ровно по~две клетки.
Докажите, что в~отмеченные клетки можно поставить $8$ чёрных и~$8$ белых ладей так, чтобы в~каждой строке и~в~каждом столбце стояла одна чёрная ладья и~одна белая ладья.

\item 
В~остроугольном треугольнике $ABC$ проведены высоты $AA_1$, $BB_1$, $CC_1$.
Докажите, что точка, симметричная точке $A_1$ относительно стороны~$AB$, лежит на~прямой~$B_1C_1$.

\item 
В~таблицу $101 \times 101$ поставлена $101$ ладья, причём ладьи не~бьют друг друга.
Каждая ладья сделала ход конём.
Могла~ли получиться расстановка ладей, вновь не~бьющих друг друга?

\item 
В~параллелограмме $ABCD$ диагональ $AC$ длиннее диагонали $BD$.
На~диагонали $AC$ отмечена такая точка $M$, что четырёхугольник $MBCD$~--- вписанный.
Докажите, что $BD$ является общей касательной окружностей, описанных вокруг треугольников $ABM$ и~$AMD$.

\item 
Мишень <<бегущий кабан>> находится в~одном из~$100$ окошек, расположенных в~ряд.
Окошки закрыты занавесками так, что для стрелка мишень все время остается невидимой.
Чтобы поразить мишень, достаточно выстрелить в~окошко, в~котором она в~момент выстрела находится.
Если мишень не~была поражена, то~сразу после выстрела она обязательно перемещается в~какое-то соседнее окошко.
Постройте алгоритм, по~которому надо стрелять, чтобы наверняка поразить мишень.

\item 
Докажите, что любое \emph{целое} число можно представить в~виде суммы пяти кубов \emph{целых} чисел.

\itemx{$^+$}
Пусть $f(x) = x^3 - x$, $g(x) = x^3 - 3 x^2 + 1$.
Докажите, что при любых действительных $\alpha$ и $\beta$, сумма которых не равна 0, многочлен $\alpha f(x) + \beta g(x)$ имеет три различных действительных корня.

\end{problems}


\clearpage\resetproblem \begingroup % \jeolmheader
    \def\jeolmdate{19 октября 2018 г.}% понедельник
    \def\jeolmauthors{Доледёнок~А.\,В., Орлов~О.\,П.}%
\jeolmheader \endgroup

\worksheet{Ортоцентр}

\noindent
В задачах листика (кроме 6-й) треугольник $ABC$ --- остроугольный и неравнобедренный; $AA_1$, $BB_1$, $CC_1$ --- его высоты, $H$ --- ортоцентр (точка пересечения высот $AA_1$, $BB_1$, $CC_1$), $O$ --- центр описанной окружности треугольника $ABC$. \\

\begin{problems}

\item
Докажите, что $H$ является точкой пересечения биссектрис треугольника $A_1B_1C_1$.

\item
\subproblem
Докажите, что $\angle{CAO} = \angle{BAH}$. \\
\subproblem
Докажите, что $OA \perp B_1C_1$.

\item 
\subproblem 
Докажите, что точка, симметричная точке $H$ относительно стороны треугольника $ABC$, лежит на описанной окружности этого треугольника. \\ 
\subproblem
Докажите, что точка $X$, симметричная точке $H$ относительно середины стороны $BC$, лежит на описанной окружности треугольника $ABC$, причём отрезок $AX$ является её диаметром.

\item 
Докажите, что расстояние от точки $O$ до стороны $BC$ вдвое меньше длины отрезка $AH$.

\item
Окружность, описанная вокруг треугольника $AB_1C_1$, вторично пересекает описанную окружность треугольника $ABC$ в точке $K$. Докажите, что прямая $KH$ делит сторону $BC$ пополам.

\item
Точки $K$, $L$, $M$, $N$ --- середины сторон $AB$, $BC$, $CD$, $AD$ соответственно вписанного четырёхугольника $ABCD$. Докажите, что ортоцентры треугольников $AKN$, $BKL$, $CML$ и $DMN$ образуют параллелограмм.

\item
Прямые $AO$ и $BC$ пересекаются в точке $Q$, а отрезки $AH$ и $B_1C_1$~--- в точке $P$. Точка $M$~--- середина $BC$. Докажите, что $HM \parallel PQ$.

\end{problems}



\clearpage\resetproblem \begingroup % \jeolmheader
    \def\jeolmdate{24 октября 2018 г.}% понедельник
    \def\jeolmauthors{Орлов~О.\,П., Тихонов~Ю.\,В.}%
\jeolmheader \endgroup

\worksheet{Неравенства о средних}

\claim{Неравенства о средних}
При любом наборе положительных чисел $a_{1}, a_{2}, \ldots, a_{n}$ верны неравенства
\[
    \sqrt{\frac{a_{1}^2 + \ldots + a_{n}^2}{n}}
\geq
    \frac{a_{1} + \ldots + a_{n}}{n}
\geq
    \sqrt[n]{a_{1} \cdot \ldots \cdot a_{n}}
\geq
    \frac{n}{\frac{1}{a_{1}} + \ldots + \frac{1}{a_{n}}}
\, . \]

Средние называются соответственно (слева направо):
\textit{квадратичное}, \textit{арифметическое}, \textit{геометрическое}, \textit{гармоническое}.

\claim{Замечание}
Во всех трёх неравенствах равенство достигается только тогда, когда все элементы набора равны между собой.

\begin{problems}

\item 
При положительных $a$, $b$, $c$ докажите неравенства
\[
    \frac{1}{a} + \frac{1}{b} + \frac{1}{c}
\geq
    \frac{2}{a + b} + \frac{2}{b + c} + \frac{2}{a + c}
\geq
    \frac{9}{a + b + c}
\, . \]

\item
При положительных $a$, $b$, $c$, $d$ докажите неравенство
\[
    \frac{1}{a^3} + \frac{1}{b^3} + \frac{1}{c^3} + \frac{1}{d^3}
\geq
    \frac{a + b + c + d}{a b c d}
\, . \]

\item 
Докажите, что
\(
    \sqrt{a + 1} + \sqrt{2 a - 3} + \sqrt{50 - 3 a} < 12
\).

\item
Положительные числа $a$, $b$, $c$ таковы, что $a b c = 1$.
Докажите, что
\[
    a^2 + b^2 + c^2
\geq
    a + b + c
\, . \]

\item 
При положительных $a$, $b$, $c$ докажите, что
\(
    a b c \geq (a + b - c) (b + c - a) (a + c - b)
\).

\item 
При положительных $x$, $y$, $z$ докажите неравенство 
\[
    x^4 + y^4 + z^2 \geq \sqrt{8} x y z
\, . \]

\item 
Положительные числа $a$, $b$, $c$, $d$ таковы, что $2 (a + b + c + d) \geq a b c d$.
Докажите, что
\[
    a^2 + b^2 + c^2 + d^2
\geq
    abcd
\, . \]

\item 
Положительные числа $a$, $b$, $c$ таковы, что $a b c = 1$.
Найдите минимальное значение выражения $2 a^3 + 3 b^2 + 6 c$.

%\item 
%При положительных $a$, $b$, $c$ докажите неравенство
%\[
%    \frac{a}{2 a + b + c} + \frac{b}{2 b + a + c} + \frac{c}{2 c + a + b}
%\leq
%    \frac{3}{4}
%\, . \]

\item 
При положительных $a, b, c$ докажите неравенство
\[
    \frac{a b}{c^2} + \frac{b c}{a^2} + \frac{a c}{b^2}
\geq
    \frac{\sqrt{a b}} c + \frac{\sqrt{b c}} a + \frac{\sqrt{a c}} b
\, . \]

\end{problems}


\clearpage\resetproblem \begingroup % \jeolmheader
    \def\jeolmdate{26 октября 2018 г.}% понедельник
    \def\jeolmauthors{Доледенок~А.\,В., Орлов~О.\,П.}%
\jeolmheader \endgroup

\worksheet{КБШ}

\claim{Неравенство Коши-Буняковского-Шварца} При любых наборах чисел \\ $a_1, a_2, \ldots, a_n$ и $b_1, b_2, \ldots, b_n$ верно неравенство
$$
(a_1^2 + a_2^2 + \ldots + a_n^2)(b_1^2 + b_2^2 + \ldots + b_n^2) \geqslant (a_1 b_1 + a_2 b_2 + \ldots + a_n b_n)^2.
$$

\begin{problems}

\item При любых $a, b, c$ докажите неравенство 
$$
a \sqrt{a^2 + c^2} + b\sqrt{b^2 + c^2} \leqslant a^2 + b^2 + c^2.
$$


\item
При помощи КБШ докажите неравенство между средним квадратичным и средним арифметическим набора неотрицательных чисел $a_1, a_2, \ldots, a_n$:
$$
\sqrt{\frac{a_1^2 + \ldots + a_n^2}n} \geqslant \frac{a_1 + \ldots + a_n}n.
$$

\item При любых $\alpha, \beta$ докажите, что
$$
\sin{\alpha} \sin{\beta} + \cos{\alpha} + \cos{\beta} \leqslant 2.
$$


\item При любых положительных $a_1, a_2, \ldots, a_n$ докажите неравенство
$$
(a_1 + a_2 + \ldots + a_n)(a_1^7 + a_2^7 + \ldots + a_n^7) \geqslant (a_1^3 + a_2^3 + \ldots + a_n^3)(a_1^5 + a_2^5 + \ldots + a_n^5).
$$

\item Даны положительные числа $a_1, a_2, \ldots, a_n, b_1, b_2, \ldots, b_n, c_1, c_2, \ldots, c_n$. Докажите неравенство
$$
(a_1 b_1 c_1 + a_2 b_2 c_2 + \ldots + a_n b_n c_n)^3 \leqslant (a_1^3 + a_2^3 + \ldots + a_n^3)(b_1^3 + b_2^3 + \ldots + b_n^3)(c_1^3 + c_2^3 + \ldots + c_n^3).
$$


\item При положительных $a_1, \ldots, a_n, b_1, \ldots, b_n$ докажите неравенство
$$
\frac{a_1^2}{b_1} + \frac{a_2^2}{b_2} + \ldots + \frac{a_n^2}{b_n} \geqslant \frac{(a_1 + a_2 + \ldots + a_n)^2}{b_1 + b_2 + \ldots + b_n}.
$$

\item При положительных $a, b, c$ докажите неравенство
$$
\frac{a}{b + c} + \frac{b}{a + c} + \frac{c}{a + b} \geqslant \frac32.
$$


\item 
Неотрицательные $a, b, c, d$ таковы, что $a + b + c + d = 4$. Докажите неравенство
$$
\frac{a}{1 + b^2 c} + \frac{b}{1 + c^2 d} + \frac{c}{1 + d^2 a} + \frac{d}{1 + a^2 b} \geqslant 2.
$$

\end{problems}



\clearpage\input{2018-11/07-graph-connectivity.tex}
\clearpage\resetproblem \begingroup % \jeolmheader
    \def\jeolmdate{9 ноября 2018 г.}% понедельник
    \def\jeolmauthors{Орлов~О.\,П.}%
\jeolmheader \endgroup

\worksheet{Графы. Деревья}

\claim{Определение} \textit{Деревом} называется связный граф без циклов. \textit{Висячей} вершиной в графе называется вершина степени $1$.

\claim{Утверждение} В любом дереве с $n \geqslant 2$ вершинами есть как минимум две висячие вершины.

\claim{Утверждение} В любом дереве с $n$ вершинами ровно $n-1$ ребро.

\claim{Определение} \textit{Остовным деревом} графа называется \textit{подграф}, содержащий все его вершины и являющийся деревом.

\claim{Утверждение} В любом связном графе можно выделить остовное дерево.

\claim{Утверждение} Если в связном графе с $n$ вершинами $n-1$ ребро, то это дерево.

\claim{Утверждение} В любом связном графе с $n$ вершинами не меньше $n-1$ ребер.

\begin{problems}

\item Докажите, что из любого связного графа можно убрать одну вершину вместе со всеми выходящими из неё рёбрами так, чтобы граф остался связным.

\item Куб $n \times n \times n$ разбит на кубики $1 \times 1 \times 1$. Какое минимальное количество граней $1 \times 1$ необходимо в нём убрать, чтобы из любой его части можно было пробраться наружу?

\item В стране $100$ городов, некоторые из которых соединены авиалиниями. Известно, что от любого города можно долететь до любого другого (возможно, с пересадками). Докажите, что можно пролететь по всем городам, сделав при этом не более $196$ перелётов.

\item Даны натуральные взаимно простые числа $p$ и $q$. Вася не знает сколько человек придёт к нему на день рождения, либо $p$, либо $q$. На какое минимальное количество кусков (не обязательно равных) ему нужно заранее разрезать торт, чтобы он мог раздать торт поровну как на $p$ человек, так и на $q$?

\item Петя поставил на доску $50\times 50$ несколько фишек, в каждую клетку -- не больше одной. Докажите, что Вася может поставить на свободные поля этой же доски не более $99$ новых фишек (возможно, ни одной) так, чтобы по-прежнему в каждой клетке стояло не больше одной фишки, и в каждой строке и каждом столбце этой доски оказалось чётное количество фишек.

\item Все $n$ вершин графа $G$ занумерованы. Известно, что множество вершин графа можно разбить на пять (потенциально пустых) долей с условием, чтобы внутри этих долей не было рёбер, причём это можно сделать \textit{единственным} с точностью до перестановки долей образом. Докажите, что в графе хотя бы $4n - 10$ рёбер.

\item Дано дерево с $n$ вершинами. В его вершинах расставлены числа $x_1, x_2, \ldots, x_n$, а на каждом ребре записано произведение чисел, стоящих в концах этого ребра. Обозначим через $S$ сумму чисел на всех рёбрах. Докажите, что $\sqrt{n-1}(x_1^2 + x_2^2 + \ldots + x_n^2) \geqslant 2S$.

\end{problems}



\clearpage\resetproblem \begingroup % \jeolmheader
    \def\jeolmdate{14 ноября 2018 г.}% среда
    \def\jeolmauthors{Орлов О.\,П., Тихонов Ю.\,В.}%
\jeolmheader \endgroup

\worksheet{Разнобой-2}

\begin{problems}

\item Уравнение $a x^5 + b x + c = 0$ имеет ровно три различных корня ($c \neq 0$). Докажите, что уравнение $c x^5 + b x^4 + a = 0$ тоже имеет ровно три различных корня.

\item Положительные числа $a, b, c$ таковы, что $ab + bc + ac \geqslant a + b + c$. Докажите, что $a + b + c \geqslant 3$.

\item Для вписанного четырехугольника выполнено равенство $CD = AD + BC$. Докажите, что точка пересечения биссектрис углов $A$ и $B$ лежит на прямой $CD$.

\item Петя и Вася играют в очень странные шашки на полоске $1 \times n$. Изначально в последних (самых правых) трёх клетках полоски стоит три фишки (по одной в клетке). За ход игрок берёт любую фишку и передвигает её налево в любую свободную клетку (ничего не мешает фишкам перепрыгивать друг друга). Проигрывает тот, кто не может сделать ход. Кто выигрывает при правильной игре, если Петя начинает первым?

\item В графе у любых двух смежных вершин ровно $5$ общих смежных вершин. Докажите, что количество рёбер в этом графе делится на $3$.

\item Дано $10$ натуральных чисел. Докажите, что можно выбрать несколько из них так, чтобы сумма выбранных делилась на $10$.

\item Дано натуральное число $n > 2$. Рассмотрим все покраски клеток доски $n \times n$ в $k$ цветов такие, что каждая клетка покрашена ровно в один цвет, и все $k$ цветов встречаются. При каком наименьшем $k$ в любой такой покраске найдутся четыре окрашенных в четыре разных цвета клетки, расположенные в пересечении двух строк и двух столбцов?

\item
Найдите углы остроугольного треугольника $ABC$, если известно, что его
биссектриса~$AD$ равна стороне~$AC$ и~перпендикулярна отрезку~$OH$, где $O$~---
центр описанной окружности, $H$~--- точка пересечения высот треугольника $ABC$.







\end{problems}


\clearpage\resetproblem \begingroup % \jeolmheader
    \def\jeolmdate{16 ноября 2018 г.}% среда
    \def\jeolmauthors{Доледенок А.\,В., Орлов О.\,П.}%
\jeolmheader \endgroup

\worksheet{Комбинаторная геометрия. Принцип крайнего}

\begin{problems}


\item Из точки $O$ выходит несколько лучей так, что угол между любыми двумя меньше $120^{\circ}$. Докажите, что среди этих лучей найдутся два луча, в угле между которыми находятся все остальные.
% Два луча с наибольшим углом между ними являются искомыми. Другой луч не может лежать в соседнем угле по построению, и не может лежать в вертикальном угле, иначе сумма углов между тремя рассмотренными лучами меньше 360.

\item Несколько прямых общего положения делят плоскость на части (никакие две прямые не параллельны, никакие три не пересекаются в одной точке). Докажите, что к каждой прямой примыкает хотя бы один треугольник.
% Возьмем ближайшую к рассматриваемой прямой точку пересечения других прямых $l_1$ и $l_2$. Эти три прямые образуют треугольник, примыкающий к рассматриваемой прямой. Никакая другая прямая не пересекает этот треугольник, так как тогда она бы пересекала хотя бы одну сторону треугольника, лежащей на $l_1$ или $l_2$, что противоречило бы выбору ближайшей точки пересечения прямых к рассматриваемой.

\item Несколько прямых общего положения делят плоскость на части. Докажите, что хотя бы одна часть является углом.
%б) Докажите, что хотя бы три части являются углами.
% Принцип крайнего по координате.
% В пункте а) достаточно взять самую левую точку пересечения прямых в системе координат, в которой оси не параллельны ни одной из данных прямых.
% Пункт б). Рассмотрим в такой системе координат самую левую, самую правую, самую верхнюю и самую нижнюю точки. Если из рассмотренных хотя бы три различных, то для каждой мы найдем свой угол и мы победили. Пусть две, от каждой из них выходит по углу. Соединим эти точки прямой $l$ и рассмотрим самую дальную от этой прямой точку пересечения данных прямых, если их несколько то возьмём из них крайнюю на прямой, их содержащей. Она и будет точкой, из которой исходит третий искомый угол. 

\item Дан выпуклый многоугольник площади $1$. \\
а) Докажите, что существует прямоугольник площади $2$, в котором данный многоугольник полностью содержится; \\
б) Докажите, что существует прямоугольник площади $1/8$, который целиком содержится в данном многоугольнике.
% Рассмотрим две вершины многоугольника $A$ и $B$ на наибольшем расстоянии. Тогда весь многоугольник лежит в полосе, в которой $AB$ --- высота. Пусть полоса находится вертикально. Возьмем самую высокую вершину многоугольника $C$ (если их несколько --- то любую из них) и самую низкую --- $D$ ($C$ или $D$ может совпадать с $A$ или $B$). Прямые, параллельные $AB$, проходящие через $C$ и $D$ ограничивают полосу до прямоугольника $P$.
% Пункт а). Прямоугольник $P$ искомый, так как $S_P = 2S_{ABC} + 2S_{ABD} \leqslant 2S_{многоуг} = 2.
% Пункт б). Заметим, что $S_P \geqslant S_{многоуг} = 1$. Прямая $AB$ разрезает $P$ на два более маленьких прямоугольника. Без ограничения общности пусть площадь верхнего, содержащего точку $C$, не меньше половины $P$, а значит не меньше $1/2$. Тогда $S_{ABC} \geqslant 1/4$. А в треугольник $ABC$ можно вписать прямоугольник $p$, площадь которого не меньше половины площади $ABC$ ($p$ содержит среднюю линюю треугольника).

\item Дан треугольник со сторонами, не превосходящими $1$. Докажите, что круги радиуса $1/\sqrt{3}$ с центрами в вершинах треугольника полностью его покрывают.
% ABC --- треугольник. Пусть точка $X$ не покрыта. Тогда $XA, XB, XC \geqslant 1 / \sqrt{3}$. Хотя бы один из углов $AXB, BXC, CXA$ не меньше 120, пусть $AXB$. Но тогда по теореме косинусов $AB > 1$, противоречие.

\item На плоскости отмечено $N \geqslant 3$ различных точек. Известно, что среди попарных расстояний между отмеченными точками встречаются не более $n$ различных расстояний. Докажите, что $N \leqslant (n + 1)^2$.
% Найдём такие точки $AB$, что все остальные лежат по одну сторону от $AB$ (возможно на $AB$ какие-то). Для этого рассмотрим самую левую точку $A$ и повращаем вертикальную прямую вокруг $A$, пока она не наткнётся на искомую точку $B$. Отмеченные точки лежат на $n$ концентрических окружностях с центрами в $A$ и на $n$ концентрических окружностях с центрами в $B$. Точек пересечения всех этих окружностей не больше $2n^2$ и не более $n^2$ в одной полуплоскости относительно $AB$. Поэтому всего отмеченных точек не больше $n^2 + 2 \leqslant (n + 1)^2$.

\item На плоскости отмечено несколько точек, причём на каждой прямой, проходящей через любые две отмеченные точки, есть ещё хотя бы одна отмеченная точка. Докажите, что все отмеченные точки лежат на одной прямой.
% Рассмотрим все пары $(P, l)$, где $P$ --- отмеченная точка, которая не лежит на прямой $l$, соединяющей две какие-то другие отмеченные точки. Пусть это множество не пусто, возьмем тогда пару с наименьшим расстоянием между элементами пары. По условию на $l$ лежит хотя бы три отмеченных точки $A, B, C$. Опустим высоту $PH$ на $l$. Тогда хотя бы две точки из $A, B, C$, по одну сторону относительно $H$, пусть это точки $A$ и $B$, причём точка $A$ лежит дальше от $H$. Тогда, очевидно, расстояние от точки $B$ до прямой $AP$ меньше, чем $PH$, противоречие.


\end{problems}


\clearpage\resetproblem \begingroup % \jeolmheader
    \def\jeolmdate{28 ноября 2018 г.}% среда
    \def\jeolmauthors{Орлов О.\,П., Тихонов Ю.\,В.}%
\jeolmheader \endgroup

\worksheet{Разнобой-3}

\begin{problems}

\item В ромбе $ABCD$ на стороне $BC$ отмечена точка $P$, а на отрезке $BD$ точки $S$ и $Q$. Оказалось, что точки $P, C, S, Q$ лежат на одной окружности и точки $A, B, P, S$ тоже лежат на одной окружности. Докажите, что точки $A, Q, P$ лежат на одной прямой.

\item Шахматная доска раскрашена в $10$ цветов правильным образом (соседние по стороне клетки
имеют разные цвета), причём все цвета присутствуют. Два цвета называются соседними, если
существуют две соседние клетки этих цветов. Каково наименьшее число пар соседних цветов?

\item
Найдите углы остроугольного треугольника $ABC$, если известно, что его
биссектриса~$AD$ равна стороне~$AC$ и~перпендикулярна отрезку~$OH$, где $O$~---
центр описанной окружности, $H$~--- точка пересечения высот треугольника $ABC$.

\item В графе у любых двух смежных вершин ровно $5$ общих смежных вершин. Докажите, что количество рёбер в этом графе делится на $3$.

\item Числа $a, b, c$ таковы, что уравнение $x^3 + ax^2 + bx + c = 0$ имеет три различных действительных корня. Докажите, что если $-2 \leqslant a + b + c \leqslant 0$, то хотя бы один из этих корней принадлежит отрезку $[0, 2]$.

\item Из Ленинграда в Москву поочерёдно выходят путники, на каждом из которых сидит своя муха.
Все путники идут с постоянными различными скоростями; для любой пары путников известно,
что более медленный из них вышел из Ленинграда раньше, но пришёл в Москву позже. Когда
два путника встречаются, мухи на них меняются местами (никакие три путника не встречаются
вместе одновременно). Докажите, что какая-то муха побывает на всех путниках.



\end{problems}




% }}}1

\tableofcontents
\let\worksheetsave\worksheet
\def\worksheet#1{\maybephantomsection\addcontentsline{toc}{section}{#1}%
    \worksheetsave{#1}}
\clearpage\input{contents.tex} % auto-generated


% vim: set foldmethod=marker :%%%%%%%%%%%%%%%%%%%%%%%%%%%%%%%%%%%%%%%%%%%%%%%%%%
