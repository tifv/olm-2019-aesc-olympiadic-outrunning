\documentclass[a4paper,10pt]{article}
\usepackage[utf8]{inputenc}

% Конфигурация — общее {{{1

\usepackage{multicol}
\usepackage[svgnames]{xcolor}
\usepackage{graphicx}

\usepackage[lazy-figures]{jeolm}
\usepackage{jeolm-groups}

\usepackage[T2A]{fontenc}
\usepackage{anyfontsize}
\usepackage{amsmath}
\usepackage{amssymb}
\usepackage{upgreek}
\AtBeginDocument{\swapvar{phi}\swapvar{epsilon}}
\AtBeginDocument{\swapvar[up]{phi}\swapvar[up]{epsilon}}
\AtBeginDocument{\let\geq\geqslant\let\leq\leqslant}
\usepackage[russian]{babel}
\usepackage{parskip}
\pagestyle{empty}

% Конфигурация — страницы {{{1

\usepackage{geometry}
% Размер «логической» страницы
\geometry{a5paper,portrait,vmargin={2em,2em},hmargin={2em,2em}}
%\geometry{a6paper,landscape,vmargin={1.5em,1.5em},hmargin={1.5em,1.5em}}

\usepackage{pgfpages}
% Размещение «логических» страниц на «физической»
%\pgfpagesuselayout{resize to}[a4paper]
\pgfpagesuselayout{2 on 1}[a4paper,landscape]
%\pgfpagesuselayout{4 on 1}[a4paper,landscape]

\usepackage{hyperref}
\hypersetup{colorlinks,urlcolor=blue}
\def\maybephantomsection{\ifdefined\phantomsection\phantomsection\fi}

% Конфигурация — локальное {{{1

\renewcommand\jeolminstitution
  {СУНЦ МГУ, Олимпиадная математика}%
\renewcommand\jeolmdaterange
  {2018--2019}%
\def\jeolmgroupname{Убегающие}%

% Конфигурация — размер шрифта {{{1

% Уменьшить основной размер шрифта
\AtBeginDocument{\fontsize{9.00}{10.80}\selectfont}
% (по умолчанию 10.00 и 12.00 соответственно)

% }}}1


\begin{document}

%\clearpage\resetproblem \begingroup % \jeolmheader
    \def\jeolmdate{26 октября 2018 г.}% понедельник
    \def\jeolmauthors{Доледенок~А.\,В., Орлов~О.\,П.}%
\jeolmheader \endgroup

\worksheet{КБШ}

\claim{Неравенство Коши-Буняковского-Шварца} При любых наборах чисел \\ $a_1, a_2, \ldots, a_n$ и $b_1, b_2, \ldots, b_n$ верно неравенство
$$
(a_1^2 + a_2^2 + \ldots + a_n^2)(b_1^2 + b_2^2 + \ldots + b_n^2) \geqslant (a_1 b_1 + a_2 b_2 + \ldots + a_n b_n)^2.
$$

\begin{problems}

\item При любых $a, b, c$ докажите неравенство 
$$
a \sqrt{a^2 + c^2} + b\sqrt{b^2 + c^2} \leqslant a^2 + b^2 + c^2.
$$


\item
При помощи КБШ докажите неравенство между средним квадратичным и средним арифметическим набора неотрицательных чисел $a_1, a_2, \ldots, a_n$:
$$
\sqrt{\frac{a_1^2 + \ldots + a_n^2}n} \geqslant \frac{a_1 + \ldots + a_n}n.
$$

\item При любых $\alpha, \beta$ докажите, что
$$
\sin{\alpha} \sin{\beta} + \cos{\alpha} + \cos{\beta} \leqslant 2.
$$


\item При любых положительных $a_1, a_2, \ldots, a_n$ докажите неравенство
$$
(a_1 + a_2 + \ldots + a_n)(a_1^7 + a_2^7 + \ldots + a_n^7) \geqslant (a_1^3 + a_2^3 + \ldots + a_n^3)(a_1^5 + a_2^5 + \ldots + a_n^5).
$$

\item Даны положительные числа $a_1, a_2, \ldots, a_n, b_1, b_2, \ldots, b_n, c_1, c_2, \ldots, c_n$. Докажите неравенство
$$
(a_1 b_1 c_1 + a_2 b_2 c_2 + \ldots + a_n b_n c_n)^3 \leqslant (a_1^3 + a_2^3 + \ldots + a_n^3)(b_1^3 + b_2^3 + \ldots + b_n^3)(c_1^3 + c_2^3 + \ldots + c_n^3).
$$


\item При положительных $a_1, \ldots, a_n, b_1, \ldots, b_n$ докажите неравенство
$$
\frac{a_1^2}{b_1} + \frac{a_2^2}{b_2} + \ldots + \frac{a_n^2}{b_n} \geqslant \frac{(a_1 + a_2 + \ldots + a_n)^2}{b_1 + b_2 + \ldots + b_n}.
$$

\item При положительных $a, b, c$ докажите неравенство
$$
\frac{a}{b + c} + \frac{b}{a + c} + \frac{c}{a + b} \geqslant \frac32.
$$


\item 
Неотрицательные $a, b, c, d$ таковы, что $a + b + c + d = 4$. Докажите неравенство
$$
\frac{a}{1 + b^2 c} + \frac{b}{1 + c^2 d} + \frac{c}{1 + d^2 a} + \frac{d}{1 + a^2 b} \geqslant 2.
$$

\end{problems}


%\clearpage\resetproblem \begingroup % \jeolmheader
    \def\jeolmdate{26 октября 2018 г.}% понедельник
    \def\jeolmauthors{Доледенок~А.\,В., Орлов~О.\,П.}%
\jeolmheader \endgroup

\worksheet{КБШ}

\claim{Неравенство Коши-Буняковского-Шварца} При любых наборах чисел \\ $a_1, a_2, \ldots, a_n$ и $b_1, b_2, \ldots, b_n$ верно неравенство
$$
(a_1^2 + a_2^2 + \ldots + a_n^2)(b_1^2 + b_2^2 + \ldots + b_n^2) \geqslant (a_1 b_1 + a_2 b_2 + \ldots + a_n b_n)^2.
$$

\begin{problems}

\item При любых $a, b, c$ докажите неравенство 
$$
a \sqrt{a^2 + c^2} + b\sqrt{b^2 + c^2} \leqslant a^2 + b^2 + c^2.
$$


\item
При помощи КБШ докажите неравенство между средним квадратичным и средним арифметическим набора неотрицательных чисел $a_1, a_2, \ldots, a_n$:
$$
\sqrt{\frac{a_1^2 + \ldots + a_n^2}n} \geqslant \frac{a_1 + \ldots + a_n}n.
$$

\item При любых $\alpha, \beta$ докажите, что
$$
\sin{\alpha} \sin{\beta} + \cos{\alpha} + \cos{\beta} \leqslant 2.
$$


\item При любых положительных $a_1, a_2, \ldots, a_n$ докажите неравенство
$$
(a_1 + a_2 + \ldots + a_n)(a_1^7 + a_2^7 + \ldots + a_n^7) \geqslant (a_1^3 + a_2^3 + \ldots + a_n^3)(a_1^5 + a_2^5 + \ldots + a_n^5).
$$

\item Даны положительные числа $a_1, a_2, \ldots, a_n, b_1, b_2, \ldots, b_n, c_1, c_2, \ldots, c_n$. Докажите неравенство
$$
(a_1 b_1 c_1 + a_2 b_2 c_2 + \ldots + a_n b_n c_n)^3 \leqslant (a_1^3 + a_2^3 + \ldots + a_n^3)(b_1^3 + b_2^3 + \ldots + b_n^3)(c_1^3 + c_2^3 + \ldots + c_n^3).
$$


\item При положительных $a_1, \ldots, a_n, b_1, \ldots, b_n$ докажите неравенство
$$
\frac{a_1^2}{b_1} + \frac{a_2^2}{b_2} + \ldots + \frac{a_n^2}{b_n} \geqslant \frac{(a_1 + a_2 + \ldots + a_n)^2}{b_1 + b_2 + \ldots + b_n}.
$$

\item При положительных $a, b, c$ докажите неравенство
$$
\frac{a}{b + c} + \frac{b}{a + c} + \frac{c}{a + b} \geqslant \frac32.
$$


\item 
Неотрицательные $a, b, c, d$ таковы, что $a + b + c + d = 4$. Докажите неравенство
$$
\frac{a}{1 + b^2 c} + \frac{b}{1 + c^2 d} + \frac{c}{1 + d^2 a} + \frac{d}{1 + a^2 b} \geqslant 2.
$$

\end{problems}


\clearpage\resetproblem \begingroup % \jeolmheader
    \def\jeolmdate{7 ноября 2018 г.}% понедельник
    \def\jeolmauthors{Орлов~О.\,П., Тихонов~Ю.\,В.}%
\jeolmheader \endgroup

\worksheet{Графы. Связность}

\claim{Определение} \textit{Граф} задан, если задано множество его \textit{вершин} и для любой пары различных вершин известно, связаны они \textit{ребром} или нет. \textit{Степенью} вершины называется количество выходящих из неё рёбер. Вершина называется \textit{чётной}, если её степень чётна и \textit{нечётной} --- если её степень нечётна. Степень вершины $v$ обозначается $\deg{v}$.

\claim{Утверждение} Количество рёбер в графе есть полусумма степеней всех его вершин.

\claim{Определение} \textit{Путём} в графе называется последовательность рёбер, каждое следующее из которых начинается в конце предыдущего. \textit{Циклом} в графе называется замкнутый путь, то есть такой, начало и конец которого совпадают. 

\claim{Определение} Граф называется \textit{связным}, если любые две его вершины могут быть соединены путём. Любой граф является объединением некоторых связных <<кусков>>, каждый такой <<кусок>> называется \textit{компонентой связности}. 

\begin{problems}

\item В стране Семерка $15$ городов, каждый из которых соединен дорогами не менее, чем с $7$ другими. Докажите, что из любого города можно добраться до любого другого (возможно, проезжая через другие города).

\item В Тридевятом царстве лишь один вид транспорта -- ковёр-самолёт. Из столицы выходит $21$ ковролиния, из города Дальний - одна, а из всех остальных городов -- по $20$. Докажите, что из столицы можно долететь в Дальний (возможно, с пересадками).

\item Между некоторыми из $2n$ городов установлено воздушное сообщение, причём каждый город связан (беспосадочными рейсами) не менее чем с $n$ другими. Докажите, что если отменить любые $n-1$ рейсов, то всё равно из любого города можно добраться в любой другой на самолётах (с пересадками).

\item Столица страны соединена авиалиниями со $100$ городами, а каждый город, кроме столицы, соединён авиалиниями ровно с $10$ городами (если $A$ соединён с $B$, то $B$ соединён с $A$). Известно, что из любого города можно попасть в любой другой (может быть, с пересадками). Докажите, что можно закрыть половину авиалиний, идущих из столицы так, что возможность попасть из любого города в любой другой сохранится.

\item Гидры состоят из голов и шей (любая шея соединяет ровно две головы). Одним ударом меча можно снести все шеи, выходящие из какой-то головы $A$ гидры. Но при этом из головы $A$ мгновенно вырастает по одной шее во все головы, с которыми $A$ не была соединена. Геракл побеждает гидру, если ему удастся разрубить её на
две несвязанные шеями части. Найдите наименьшее $N$, при котором Геракл сможет победить любую  \textit{стошеюю} гидру, нанеся не более, чем $N$ ударов.

\item В стране $1001$ город, любые два города соединены дорогой с односторонним движением (граф в котором на каждом ребре задано направление называется \textit{ориентированным}). Из каждого города выходит ровно $500$ дорог, в каждый город входит ровно $500$ дорог. От страны отделилась независимая республика, в которую вошли $668$ городов. Докажите, что из любого города этой республики можно доехать до любого другого её города, не выезжая за пределы республики.

\end{problems}



\clearpage\resetproblem \begingroup % \jeolmheader
    \def\jeolmdate{7 ноября 2018 г.}% понедельник
    \def\jeolmauthors{Орлов~О.\,П., Тихонов~Ю.\,В.}%
\jeolmheader \endgroup

\worksheet{Графы. Связность}

\claim{Определение} \textit{Граф} задан, если задано множество его \textit{вершин} и для любой пары различных вершин известно, связаны они \textit{ребром} или нет. \textit{Степенью} вершины называется количество выходящих из неё рёбер. Вершина называется \textit{чётной}, если её степень чётна и \textit{нечётной} --- если её степень нечётна. Степень вершины $v$ обозначается $\deg{v}$.

\claim{Утверждение} Количество рёбер в графе есть полусумма степеней всех его вершин.

\claim{Определение} \textit{Путём} в графе называется последовательность рёбер, каждое следующее из которых начинается в конце предыдущего. \textit{Циклом} в графе называется замкнутый путь, то есть такой, начало и конец которого совпадают. 

\claim{Определение} Граф называется \textit{связным}, если любые две его вершины могут быть соединены путём. Любой граф является объединением некоторых связных <<кусков>>, каждый такой <<кусок>> называется \textit{компонентой связности}. 

\begin{problems}

\item В стране Семерка $15$ городов, каждый из которых соединен дорогами не менее, чем с $7$ другими. Докажите, что из любого города можно добраться до любого другого (возможно, проезжая через другие города).

\item В Тридевятом царстве лишь один вид транспорта -- ковёр-самолёт. Из столицы выходит $21$ ковролиния, из города Дальний - одна, а из всех остальных городов -- по $20$. Докажите, что из столицы можно долететь в Дальний (возможно, с пересадками).

\item Между некоторыми из $2n$ городов установлено воздушное сообщение, причём каждый город связан (беспосадочными рейсами) не менее чем с $n$ другими. Докажите, что если отменить любые $n-1$ рейсов, то всё равно из любого города можно добраться в любой другой на самолётах (с пересадками).

\item Столица страны соединена авиалиниями со $100$ городами, а каждый город, кроме столицы, соединён авиалиниями ровно с $10$ городами (если $A$ соединён с $B$, то $B$ соединён с $A$). Известно, что из любого города можно попасть в любой другой (может быть, с пересадками). Докажите, что можно закрыть половину авиалиний, идущих из столицы так, что возможность попасть из любого города в любой другой сохранится.

\item Гидры состоят из голов и шей (любая шея соединяет ровно две головы). Одним ударом меча можно снести все шеи, выходящие из какой-то головы $A$ гидры. Но при этом из головы $A$ мгновенно вырастает по одной шее во все головы, с которыми $A$ не была соединена. Геракл побеждает гидру, если ему удастся разрубить её на
две несвязанные шеями части. Найдите наименьшее $N$, при котором Геракл сможет победить любую  \textit{стошеюю} гидру, нанеся не более, чем $N$ ударов.

\item В стране $1001$ город, любые два города соединены дорогой с односторонним движением (граф в котором на каждом ребре задано направление называется \textit{ориентированным}). Из каждого города выходит ровно $500$ дорог, в каждый город входит ровно $500$ дорог. От страны отделилась независимая республика, в которую вошли $668$ городов. Докажите, что из любого города этой республики можно доехать до любого другого её города, не выезжая за пределы республики.

\end{problems}





\end{document}

% Архив {{{1

\clearpage\resetproblem \begingroup % \jeolmheader
    \def\jeolmdate{17 октября 2018 г.}% среда
    \def\jeolmauthors{Орлов О.\,П., Тихонов Ю.\,В.}%
\jeolmheader \endgroup

\worksheet{Разнобой}

\begin{problems}

\item 
При любых действительных $a$, $b$, $c$ докажите неравенство
\[
    a^2 b^2 + b^2 c^2 + c^2 a^2 \geq abc(a + b + c)
\, . \]

\item 
На~доске $8 \times 8$ отмечено $16$ клеток, причём в~каждой строке и~в~каждом столбце отмечено ровно по~две клетки.
Докажите, что в~отмеченные клетки можно поставить $8$ чёрных и~$8$ белых ладей так, чтобы в~каждой строке и~в~каждом столбце стояла одна чёрная ладья и~одна белая ладья.

\item 
В~остроугольном треугольнике $ABC$ проведены высоты $AA_1$, $BB_1$, $CC_1$.
Докажите, что точка, симметричная точке $A_1$ относительно стороны~$AB$, лежит на~прямой~$B_1C_1$.

\item 
В~таблицу $101 \times 101$ поставлена $101$ ладья, причём ладьи не~бьют друг друга.
Каждая ладья сделала ход конём.
Могла~ли получиться расстановка ладей, вновь не~бьющих друг друга?

\item 
В~параллелограмме $ABCD$ диагональ $AC$ длиннее диагонали $BD$.
На~диагонали $AC$ отмечена такая точка $M$, что четырёхугольник $MBCD$~--- вписанный.
Докажите, что $BD$ является общей касательной окружностей, описанных вокруг треугольников $ABM$ и~$AMD$.

\item 
Мишень <<бегущий кабан>> находится в~одном из~$100$ окошек, расположенных в~ряд.
Окошки закрыты занавесками так, что для стрелка мишень все время остается невидимой.
Чтобы поразить мишень, достаточно выстрелить в~окошко, в~котором она в~момент выстрела находится.
Если мишень не~была поражена, то~сразу после выстрела она обязательно перемещается в~какое-то соседнее окошко.
Постройте алгоритм, по~которому надо стрелять, чтобы наверняка поразить мишень.

\item 
Докажите, что любое \emph{целое} число можно представить в~виде суммы пяти кубов \emph{целых} чисел.

\itemx{$^+$}
Пусть $f(x) = x^3 - x$, $g(x) = x^3 - 3 x^2 + 1$.
Докажите, что при любых действительных $\alpha$ и $\beta$, сумма которых не равна 0, многочлен $\alpha f(x) + \beta g(x)$ имеет три различных действительных корня.

\end{problems}


\clearpage\resetproblem \begingroup % \jeolmheader
    \def\jeolmdate{19 октября 2018 г.}% понедельник
    \def\jeolmauthors{Доледёнок~А.\,В., Орлов~О.\,П.}%
\jeolmheader \endgroup

\worksheet{Ортоцентр}

\noindent
В задачах листика (кроме $6$-й) треугольник $ABC$ --- остроугольный и неравнобедренный; $AA_1$, $BB_1$, $CC_1$ --- его высоты, $H$ --- ортоцентр (точка пересечения высот $AA_1$, $BB_1$, $CC_1$), $O$ --- центр описанной окружности треугольника $ABC$. \\

\begin{problems}

\item
Докажите, что $H$ является точкой пересечения биссектрис треугольника $A_1B_1C_1$.

\item
\subproblem
Докажите, что $\angle{CAO} = \angle{BAH}$. \\
\subproblem
Докажите, что $OA \perp B_1C_1$.

\item 
\subproblem 
Докажите, что точка, симметричная точке $H$ относительно стороны треугольника $ABC$, лежит на описанной окружности этого треугольника. \\ 
\subproblem
Докажите, что точка $X$, симметричная точке $H$ относительно середины стороны $BC$, лежит на описанной окружности треугольника $ABC$, причём отрезок $AX$ является её диаметром.

\item 
Докажите, что расстояние от точки $O$ до стороны $BC$ вдвое меньше длины отрезка $AH$.

\item
Окружность, описанная вокруг треугольника $AB_1C_1$, вторично пересекает описанную окружность треугольника $ABC$ в точке $K$. Докажите, что прямая $KH$ делит сторону $BC$ пополам.

\item
Точки $K$, $L$, $M$, $N$ --- середины сторон $AB$, $BC$, $CD$, $AD$ соответственно вписанного четырёхугольника $ABCD$. Докажите, что ортоцентры треугольников $AKN$, $BKL$, $CML$ и $DMN$ образуют параллелограмм.

\item
Прямые $AO$ и $BC$ пересекаются в точке $Q$, а отрезки $AH$ и $B_1C_1$~--- в точке $P$. Точка $M$~--- середина $BC$. Докажите, что $HM \parallel PQ$.

\end{problems}



\clearpage\resetproblem \begingroup % \jeolmheader
    \def\jeolmdate{24 октября 2018 г.}% понедельник
    \def\jeolmauthors{Орлов~О.\,П., Тихонов~Ю.\,В.}%
\jeolmheader \endgroup

\worksheet{Неравенства о средних}

\claim{Неравенства о средних}
При любом наборе положительных чисел $a_{1}, a_{2}, \ldots, a_{n}$ верны неравенства
\[
    \sqrt{\frac{a_{1}^2 + \ldots + a_{n}^2}{n}}
\geq
    \frac{a_{1} + \ldots + a_{n}}{n}
\geq
    \sqrt[n]{a_{1} \cdot \ldots \cdot a_{n}}
\geq
    \frac{n}{\frac{1}{a_{1}} + \ldots + \frac{1}{a_{n}}}
\, . \]

Средние называются соответственно (слева направо):
\textit{квадратичное}, \textit{арифметическое}, \textit{геометрическое}, \textit{гармоническое}.

\claim{Замечание}
Во всех трёх неравенствах равенство достигается только тогда, когда все элементы набора равны между собой.

\begin{problems}

\item 
При положительных $a$, $b$, $c$ докажите неравенства
\[
    \frac{1}{a} + \frac{1}{b} + \frac{1}{c}
\geq
    \frac{2}{a + b} + \frac{2}{b + c} + \frac{2}{a + c}
\geq
    \frac{9}{a + b + c}
\, . \]

\item
При положительных $a$, $b$, $c$, $d$ докажите неравенство
\[
    \frac{1}{a^3} + \frac{1}{b^3} + \frac{1}{c^3} + \frac{1}{d^3}
\geq
    \frac{a + b + c + d}{a b c d}
\, . \]

\item 
Докажите, что
\(
    \sqrt{a + 1} + \sqrt{2 a - 3} + \sqrt{50 - 3 a} < 12
\).

\item
Положительные числа $a$, $b$, $c$ таковы, что $a b c = 1$.
Докажите, что
\[
    a^2 + b^2 + c^2
\geq
    a + b + c
\, . \]

\item 
При положительных $a$, $b$, $c$ докажите, что
\(
    a b c \geq (a + b - c) (b + c - a) (a + c - b)
\).

\item 
При положительных $x$, $y$, $z$ докажите неравенство 
\[
    x^4 + y^4 + z^2 \geq \sqrt{8} \cdot x y z
\, . \]

\item 
Положительные числа $a$, $b$, $c$, $d$ таковы, что $2 (a + b + c + d) \geq a b c d$.
Докажите, что
\[
    a^2 + b^2 + c^2 + d^2
\geq
    a b c d
\, . \]

\item 
Положительные числа $a$, $b$, $c$ таковы, что $a b c = 1$.
Найдите минимальное значение выражения $2 a^3 + 3 b^2 + 6 c$.

%\item 
%При положительных $a$, $b$, $c$ докажите неравенство
%\[
%    \frac{a}{2 a + b + c} + \frac{b}{2 b + a + c} + \frac{c}{2 c + a + b}
%\leq
%    \frac{3}{4}
%\, . \]

\item 
При положительных $a, b, c$ докажите неравенство
\[
    \frac{a b}{c^2} + \frac{b c}{a^2} + \frac{a c}{b^2}
\geq
    \frac{\sqrt{a b}} c + \frac{\sqrt{b c}} a + \frac{\sqrt{a c}} b
\, . \]

\end{problems}


\clearpage\resetproblem \begingroup % \jeolmheader
    \def\jeolmdate{26 октября 2018 г.}% понедельник
    \def\jeolmauthors{Доледенок~А.\,В., Орлов~О.\,П.}%
\jeolmheader \endgroup

\worksheet{КБШ}

\claim{Неравенство Коши-Буняковского-Шварца} При любых наборах чисел \\ $a_1, a_2, \ldots, a_n$ и $b_1, b_2, \ldots, b_n$ верно неравенство
$$
(a_1^2 + a_2^2 + \ldots + a_n^2)(b_1^2 + b_2^2 + \ldots + b_n^2) \geqslant (a_1 b_1 + a_2 b_2 + \ldots + a_n b_n)^2.
$$

\begin{problems}

\item При любых $a, b, c$ докажите неравенство 
$$
a \sqrt{a^2 + c^2} + b\sqrt{b^2 + c^2} \leqslant a^2 + b^2 + c^2.
$$


\item
При помощи КБШ докажите неравенство между средним квадратичным и средним арифметическим набора неотрицательных чисел $a_1, a_2, \ldots, a_n$:
$$
\sqrt{\frac{a_1^2 + \ldots + a_n^2}n} \geqslant \frac{a_1 + \ldots + a_n}n.
$$

\item При любых $\alpha, \beta$ докажите, что
$$
\sin{\alpha} \sin{\beta} + \cos{\alpha} + \cos{\beta} \leqslant 2.
$$


\item При любых положительных $a_1, a_2, \ldots, a_n$ докажите неравенство
$$
(a_1 + a_2 + \ldots + a_n)(a_1^7 + a_2^7 + \ldots + a_n^7) \geqslant (a_1^3 + a_2^3 + \ldots + a_n^3)(a_1^5 + a_2^5 + \ldots + a_n^5).
$$

\item Даны положительные числа $a_1, a_2, \ldots, a_n, b_1, b_2, \ldots, b_n, c_1, c_2, \ldots, c_n$. Докажите неравенство
$$
(a_1 b_1 c_1 + a_2 b_2 c_2 + \ldots + a_n b_n c_n)^3 \leqslant (a_1^3 + a_2^3 + \ldots + a_n^3)(b_1^3 + b_2^3 + \ldots + b_n^3)(c_1^3 + c_2^3 + \ldots + c_n^3).
$$


\item При положительных $a_1, \ldots, a_n, b_1, \ldots, b_n$ докажите неравенство
$$
\frac{a_1^2}{b_1} + \frac{a_2^2}{b_2} + \ldots + \frac{a_n^2}{b_n} \geqslant \frac{(a_1 + a_2 + \ldots + a_n)^2}{b_1 + b_2 + \ldots + b_n}.
$$

\item При положительных $a, b, c$ докажите неравенство
$$
\frac{a}{b + c} + \frac{b}{a + c} + \frac{c}{a + b} \geqslant \frac32.
$$


\item 
Неотрицательные $a, b, c, d$ таковы, что $a + b + c + d = 4$. Докажите неравенство
$$
\frac{a}{1 + b^2 c} + \frac{b}{1 + c^2 d} + \frac{c}{1 + d^2 a} + \frac{d}{1 + a^2 b} \geqslant 2.
$$

\end{problems}



\clearpage\resetproblem \begingroup % \jeolmheader
    \def\jeolmdate{7 ноября 2018 г.}% понедельник
    \def\jeolmauthors{Орлов~О.\,П., Тихонов~Ю.\,В.}%
\jeolmheader \endgroup

\worksheet{Графы. Связность}

\claim{Определение} \textit{Граф} задан, если задано множество его \textit{вершин} и для любой пары различных вершин известно, связаны они \textit{ребром} или нет. \textit{Степенью} вершины называется количество выходящих из неё рёбер. Вершина называется \textit{чётной}, если её степень чётна и \textit{нечётной} --- если её степень нечётна. Степень вершины $v$ обозначается $\deg{v}$.

\claim{Утверждение} Количество рёбер в графе есть полусумма степеней всех его вершин.

\claim{Определение} \textit{Путём} в графе называется последовательность рёбер, каждое следующее из которых начинается в конце предыдущего. \textit{Циклом} в графе называется замкнутый путь, то есть такой, начало и конец которого совпадают. 

\claim{Определение} Граф называется \textit{связным}, если любые две его вершины могут быть соединены путём. Любой граф является объединением некоторых связных <<кусков>>, каждый такой <<кусок>> называется \textit{компонентой связности}. 

\begin{problems}

\item В стране Семерка $15$ городов, каждый из которых соединен дорогами не менее, чем с $7$ другими. Докажите, что из любого города можно добраться до любого другого (возможно, проезжая через другие города).

\item В Тридевятом царстве лишь один вид транспорта -- ковёр-самолёт. Из столицы выходит $21$ ковролиния, из города Дальний - одна, а из всех остальных городов -- по $20$. Докажите, что из столицы можно долететь в Дальний (возможно, с пересадками).

\item Между некоторыми из $2n$ городов установлено воздушное сообщение, причём каждый город связан (беспосадочными рейсами) не менее чем с $n$ другими. Докажите, что если отменить любые $n-1$ рейсов, то всё равно из любого города можно добраться в любой другой на самолётах (с пересадками).

\item Столица страны соединена авиалиниями со $100$ городами, а каждый город, кроме столицы, соединён авиалиниями ровно с $10$ городами (если $A$ соединён с $B$, то $B$ соединён с $A$). Известно, что из любого города можно попасть в любой другой (может быть, с пересадками). Докажите, что можно закрыть половину авиалиний, идущих из столицы так, что возможность попасть из любого города в любой другой сохранится.

\item Гидры состоят из голов и шей (любая шея соединяет ровно две головы). Одним ударом меча можно снести все шеи, выходящие из какой-то головы $A$ гидры. Но при этом из головы $A$ мгновенно вырастает по одной шее во все головы, с которыми $A$ не была соединена. Геракл побеждает гидру, если ему удастся разрубить её на
две несвязанные шеями части. Найдите наименьшее $N$, при котором Геракл сможет победить любую  \textit{стошеюю} гидру, нанеся не более, чем $N$ ударов.

\item В стране $1001$ город, любые два города соединены дорогой с односторонним движением (граф в котором на каждом ребре задано направление называется \textit{ориентированным}). Из каждого города выходит ровно $500$ дорог, в каждый город входит ровно $500$ дорог. От страны отделилась независимая республика, в которую вошли $668$ городов. Докажите, что из любого города этой республики можно доехать до любого другого её города, не выезжая за пределы республики.

\end{problems}





% }}}1

\tableofcontents
\let\worksheetsave\worksheet
\def\worksheet#1{\maybephantomsection\addcontentsline{toc}{section}{#1}%
    \worksheetsave{#1}}
\clearpage\input{contents.tex} % auto-generated


% vim: set foldmethod=marker :%%%%%%%%%%%%%%%%%%%%%%%%%%%%%%%%%%%%%%%%%%%%%%%%%%
